\documentclass{article}%
\setlength{\textwidth}{7.0in}
\setlength{\oddsidemargin}{-0.35in}
\setlength{\topmargin}{-0.5in}
\setlength{\textheight}{9.0in}
\setlength{\parindent}{0.3in}

\usepackage[T1]{fontenc}
\usepackage[procnames]{listings}
\usepackage{xcolor}

\begin{document}

\begin{flushright}
Western Washington University \\
\end{flushright}

\begin{center}
Problem 2
\end{center}

The solution is: \\
\begin{center}
Stolen art is secured we will meet when amelia has safely left the country with the goods time and location will be sent in a seperatate message do not I repeat do not reveal the code word to mickey we believe him to be working for the police but do not expose our suspicion we will change our enciphering technique in future messages to use the ragbaby cipher.
\end{center}

We decided that the letter was encrypted with a simple substitution cipher because we couldn't think of anything else that would generate such nonsense as was seen on the paper.
We first assumed that \%p< was "the"
We then found several other words that helped us find the rest, they were: I, we, and will.
We also used http://visca.com/regexdict/ as a tool to find words that match letter patterns.

\end{document}
